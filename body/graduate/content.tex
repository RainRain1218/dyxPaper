\chapter{绪论}

\section{研究背景及意义}

近年来,随着硬件能力的提升和相关软件框架的成熟,深度学习技术得到了迅速的发展,其中就包括深度伪造(Deepfake)技术。深度伪造通常指的是使用深度学习技术将目标人物的人脸替换到其他人的图片或视频中,从而生成该目标人物完成特定表情动作的伪造图片或视频的技术,也称为换脸技术(Face Swapping)\cite{nguyen2022deep}。此外,深度伪造还包括属性编辑(attribute manipulation)、全脸合成(entire face synthesis)\cite{tolosana2020deepfakes}和人脸表情活化(Face Reenactment)\cite{hsu2022dual}等。  

深度伪造技术在影视制作、游戏开发等娱乐服务行业有多种积极应用,受到了学术界和工业界的广泛关注。一方面,在计算机视觉等领域的顶级学术会议中不断出现新的深度伪造生成技术\cite{hsu2022dual, chen2020simswap, li2020advancing, zhu2021one, xu2022high},伪造人脸图片和视频的质量逐渐提高,视觉效果越来越逼真。另一方面,也出现了多个例如DeepFaceLab\cite{Deepfacelab}等集成式的深度伪造生成软件和Zao APP等便捷的换脸制作工具,使深度伪造技术的操作难度降低,更容易普及。
然而,深度伪造技术在发挥积极作用的同时,也带来了严重的安全隐患。首先,一些恶意攻击者可以利用深度伪造换脸技术制造和传播虚假视频,损害个人名誉、破坏新闻真实性\cite{李旭嵘2020}。由于深度伪造的人脸伪造效果越来越逼真,仅从视觉上难以进行甄别,如果缺乏有效的检测技术,将对舆论安全造成极大的威胁。其次,深度伪造技术还可以攻击现有的人脸认证系统,对个人的信息、财产安全造成危害。人脸识别认证目前已广泛使用在身份认证、移动支付等与安全高度相关的应用中,如果攻击者使用深度伪造生成视频绕过人脸识别系统,则会给这些应用的安全带来影响。已有论文发现常用的人脸识别、活体检测等API无法防御深度伪造攻击\cite{tariq2022real, li2022seeing}。因此,研究可靠的深度伪造检测技术对保证人脸认证相关应用的安全是十分必要的。

为解决深度伪造技术带来的安全隐患,深度伪造图像和视频的检测技术研究受到了学术界的持续关注。数据集方面,相继有工作构造了多个包含真实人脸和伪造人脸视频的深度伪造数据集,用于支撑检测技术的研究,其中最常用的有FaceForensics++\cite{rossler2019faceforensics++}、CelebDF\cite{li2020celeb}、DeeperForensics-1.0\cite{jiang2020deeperforensics}和FFIW\cite{zhou2021face}等。在深度伪造数据集的基础上,研究者们提出了多种检测伪造人脸的算法,在数据集上实现了较高的真伪分类准确率。其中主流的方法是研究伪造人脸和真实人脸的区别特征,例如脸部纹理异常、频域特征差异等\cite{li2020face, qian2020thinking, haliassos2021lips},或设计新的网络模型提升真伪分类的效果\cite{rossler2019faceforensics++, afchar2018mesonet, wang2022m2tr}。但是这些方法在用于实际的深度伪造检测时,还存在鲁棒性较差的问题,导致难以用于安全要求较高的场景\cite{dong2022protecting}。第一,现有检测方法普遍在新出现的伪造方法上泛化性较差。因为上述方法都需要在某个深度伪造数据集上训练真伪分类模型,所以难以避免过拟合问题,导致对训练集不包含的,新出现的更加真实的伪造人脸容易产生误判。而实际场景中,深度伪造生成技术在不断更新优化,这就要求检测技术对不同伪造方法生成的伪造人脸都能鲁棒地进行检测。第二,现有检测方法普遍难以检测经过压缩的伪造人脸。这是因为它们一般利用纹理差异等细微的特征区分真伪,而这些特征在经过压缩后的低质量图像视频中无法被检测到。在实际场景中,伪造图像和视频在互联网上传播时,通常会经过包括压缩在内的各种图像后处理,因此,能在低分辨率场景下保持鲁棒的检测方法是十分必要的。最后,现有检测方法基本都使用了深度学习模型作为分类器,而深度学习模型容易受到对抗样本攻击\cite{szegedy2013intriguing},而深度伪造检测也属于攻防场景,制作伪造人脸的攻击者可以使用对抗样本攻击来绕过检测器\cite{hussain2021adversarial}。所以,深度伪造检测研究应该充分考虑对抗场景的防御。

除鲁棒性较差的问题外,现有的深度伪造检测方法大多以通用的伪造检测为目标,很少针对指定人物的伪造检测场景设计优化方案。现实中,名人的人脸图片容易在互联网上获得,而且被伪造图像和视频时的影响范围更广泛,因此更容易成为深度伪造的对象。另一方面,当检测的范围限定在检测者关心的重点人物集合内时,检测者可以利用重点人物的真实人脸辅助检测\cite{dong2022protecting}。因此,针对重点人物保护场景设计深度伪造检测方法,既能利用辅助人脸的信息提升检测方法的鲁棒性,同时又更符合实际的深度伪造检测需求。

综上,日渐成熟的深度伪造生成技术对个人的声誉财产安全和公共舆论安全都形成了隐患,而现有的深度伪造检测在实际应用场景下存在鲁棒性较差的问题,难以保证用于实际检测时的有效性,因此亟需提升深度伪造检测方法鲁棒性的研究。本文针对更符合实际场景的重点人物深度伪造图像检测问题,研究了利用伪造人脸的身份特征融合特性进行检测的算法,提升了该场景下深度伪造检测的鲁棒性,对深度伪造检测的落地应用具有重要的意义。

\section{国内外研究现状}

我们可以用includegraphics来插入现有的jpg等格式的图片,
如\autoref{fig:zju-logo}所示。

\begin{figure}[htbp]
    \centering
    \includegraphics[width=.3\linewidth]{logo/zju}
    \caption{\label{fig:zju-logo}浙江大学LOGO}
\end{figure}


\subsection{深度伪造生成技术}

\subsection{深度伪造检测技术}
\subsubsection{通用深度伪造检测}

\subsubsection{面向重点人物场景的深度伪造检测}

\par 如\autoref{tab:sample}所示,这是一张自动调节列宽的表格。

\begin{table}[htbp]
    \caption{\label{tab:sample}自动调节列宽的表格}
    \begin{tabularx}{\linewidth}{c|X<{\centering}}
        \hline
        第一列 & 第二列 \\ \hline
        xxx & xxx \\ \hline
        xxx & xxx \\ \hline
        xxx & xxx \\ \hline
    \end{tabularx}
\end{table}


\par 如\autoref{equ:sample},这是一个公式

\begin{equation}
    \label{equ:sample}
    A=\overbrace{(a+b+c)+\underbrace{i(d+e+f)}_{\text{虚数}}}^{\text{复数}}
\end{equation}

\section{研究目的与内容}

\section{论文结构安排}

\chapter{另一章}


\begin{figure}[htbp]
    \centering
    \includegraphics[width=.3\linewidth]{example-image-a}
    \caption{\label{fig:fig-placeholder}图片占位符}
\end{figure}

\chapter{再一章}

\par 如\autoref{alg:sample},这是一个算法

\begin{algorithm}[H]
    \begin{algorithmic} % enter the algorithmic environment
        \REQUIRE $n \geq 0 \vee x \neq 0$
        \ENSURE $y = x^n$
        \STATE $y \Leftarrow 1$
        \IF{$n < 0$}
            \STATE $X \Leftarrow 1 / x$
            \STATE $N \Leftarrow -n$
        \ELSE
            \STATE $X \Leftarrow x$
            \STATE $N \Leftarrow n$
        \ENDIF
        \WHILE{$N \neq 0$}
            \IF{$N$ is even}
                \STATE $X \Leftarrow X \times X$
                \STATE $N \Leftarrow N / 2$
            \ELSE[$N$ is odd]
                \STATE $y \Leftarrow y \times X$
                \STATE $N \Leftarrow N - 1$
            \ENDIF
        \ENDWHILE
    \end{algorithmic}
    \caption{\label{alg:sample}算法样例}
\end{algorithm}